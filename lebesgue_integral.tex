\documentclass[dvipdfmx]{jsreport}
\usepackage{amsmath,amssymb}
\usepackage{enumerate}

\newtheorem{theo}{定理}[section]
\newtheorem{defi}[theo]{定義}
\newtheorem{lemm}[theo]{補題}
\newtheorem{rema}[theo]{注}
\newtheorem{corr}[theo]{系}
\newtheorem{exam}[theo]{例}
\newtheorem{prop}[theo]{命題}
\newtheorem{prob}[theo]{問題}

\makeatletter
\@addtoreset{equation}{section}
\def\theequation{\thesection.\arabic{equation}}
\makeatother

\usepackage[top=30truemm,bottom=30truemm,left=25truemm,right=25truemm]{geometry}

\begin{document}

\title{ルベーグ積分}
\author{Archaea}
\maketitle

\chapter{測度}

\section{有限加法的測度}

\begin{defi}
    (有限加法族) \\
    与えられた空間 $X$ の部分集合の族 $\mathfrak{F}$ が
    \begin{enumerate}[(i)]
        \item $\emptyset \in \mathfrak{F}$
        \item $A \in \mathfrak{F} \Rightarrow A^c \in \mathfrak{F}$
        \item $A, B \in \mathfrak{F} \Rightarrow A \cup B \in \mathfrak{F}$
    \end{enumerate}
    なる三つの条件を満たすとき, $\mathfrak{F}$ を\textgt{有限加法族}という. 
\end{defi}
\begin{rema}
    この三つの性質から以下の性質が導かれる. 
    \begin{enumerate}[(i)]
        \item $X \in \mathfrak{F}$
        \item $\mathfrak{F}$ に属する集合の和, 差, 交わりをとる演算を有限回行って得られる集合は $\mathfrak{F}$ に属する. 
    \end{enumerate}
\end{rema}

\begin{theo}
    $Z = X \times Y$ (直積空間)とし, $\mathfrak{E}, \mathfrak{F}$ をそれぞれ $X, Y$ の部分集合の有限加法族とする, $Z$ の部分集合で
    \begin{equation}
        K = E \times F \qquad (E \in \mathfrak{E}, F \in \mathfrak{F})
    \end{equation}
    なる形の集合の有限個の直和として表されるものの全体 $\mathfrak{R}$ は有限加法族である. 
\end{theo}

\begin{defi}
    (有限加法的測度) \\
    空間 $X$ とその部分集合の有限加法族 $\mathfrak{F}$ があって $\mathfrak{F}$-集合関数 $m(A)$ が
    \begin{enumerate}[(i)]
        \item 全ての $A \in \mathfrak{F}$ に対して $0 \leq m(A) \leq \infty$, 特に $m(\emptyset) = 0$
        \item $A, B \in \mathfrak{F}$, $A \cap B = \emptyset \Rightarrow m(A + B) = m(A) + m(B)$ 
    \end{enumerate}
    を満たすとき, $m$ を $(\mathfrak{F}の上の)$ \textgt{有限加法的測度}という. 
\end{defi}
\begin{rema}
    上二つの性質から
    \begin{enumerate}[(i)]
        \item (有限加法性) $\displaystyle A_1, \dots, A_n \in \mathfrak{F}, A_j \cap A_k = \emptyset \ (j \neq k) \Rightarrow m\left(\sum_{j = 1}^n A_j\right) = \sum_{j = 1}^n m(A_j)$
        \item (単調性) $A, B \in \mathfrak{F}, A \supset B \Rightarrow m(A) \geq m(B)$, 特に $m(B) < \infty \Rightarrow m(A - B) = m(A) - m(B)$
        \item (有限劣加法性) $\displaystyle A_1, \dots, A_n \in \mathfrak{F} \Rightarrow m\left( \bigcup_{j = 1}^n A_j \right) \leq \sum_{j = 1}^n m(A_j)$
    \end{enumerate}
\end{rema}

\begin{defi}
    (完全加法的な測度) \\
    有限加法族 $\mathfrak{F}$ の上の有限加法的測度 $m$ が条件 \\
    $A_1, A_2, \cdots \in \mathfrak{F}$ (可算無限個), $A_j \cap A_k = \emptyset (j \neq k)$ のとき, $A = \sum_{n = 1}^\infty A_n \in \mathfrak{F} \Rightarrow m(A) = \sum_{n = 1}^\infty m(A_n)$ となる \\
    を満たすとき, $m$ を有限加法族 $\mathfrak{F}$ の上で\textgt{完全加法的}な測度という. 
\end{defi}

\begin{exam} \label{finitely_additive_measure_example}
    $X = R^N, \mathfrak{F} = \mathfrak{F}_N$ とし, $f_1(\lambda), \dots, f_N(\lambda)$ を $R^1$ で単調増加な実数値関数で定数でないものとし, 有界な区間 $I = (a_1, b_1] \times \cdots \times (a_N, b_N] (-\infty < a_v < b_v < \infty)$ に対して
    \begin{equation}
        m(I) = \prod_{v = 1}^N \{f_v(b_v) - f_v(a_v)\}
    \end{equation}
    有界でない区間 $I$ に対しては
    \begin{equation}
        m(I) = \sup\{m(J) ; J は I に含まれる任意の有界区間\}
    \end{equation}
    と $m$ を定義し, 空集合 $\emptyset$ に対しては $m(\emptyset) = 0$, 区間塊 $E = I_1 + \cdots + I_n$ に対しては
    \begin{equation}
        m(E) = m(I_1) + \cdots + m(I_n)
    \end{equation}
    と定義する. この $m$ は $\mathfrak{F}_N$ の上の有限加法的測度である. 
\end{exam}

\begin{theo}
    上の例\ref{finitely_additive_measure_example}の $m$ が $\mathfrak{F}_N$ の上で完全加法的であるための必要十分条件は全ての $f_v(\lambda)$ が右連続なことである. 
\end{theo}

\begin{theo}
    $f_v(\lambda) = \lambda (v = 1, \dots, N)$ であることと, 対応する $m$ が次の二つの条件を満たすことは同値である. 
    \begin{enumerate}[(i)]
        \item $I_0 = (0, 1] \times \cdots \times (0, 1]$ (単位立方体)に対して $m(I_0) = 1$
        \item 集合 $E \in \mathfrak{F}_N$ をベクトル $x$ だけ平行移動したものを $[E + x]$ と書くとき $m([E + x]) = m(E)$
    \end{enumerate}
\end{theo}

\section{外測度}

\begin{defi}
    (外測度) \\
    空間 $X$ の全ての部分集合 $A$ に対して定義された集合関数 $\Gamma(A)$ があって
    \begin{enumerate}[(i)]
        \item (非負性) $0 \leq \Gamma(A) \leq \infty$
        \item (単調性) $A \subset B \Rightarrow \Gamma(A) \leq \Gamma(B)$
        \item (劣加法性) $\displaystyle \Gamma\left( \bigcup_{n = 1}^\infty A_n \right) \leq \sum_{n = 1}^\infty \Gamma(A_n)$
    \end{enumerate}
    なる三つの条件を満たすとき,  $\Gamma$ を{\bf Carathéodory}\textgt{外測度}もしくは単に\textgt{外測度}という. 
\end{defi}

\begin{theo} \label{outer_measure}
    $\mathfrak{F}$ を $X$ の部分集合の有限加法族とし, $m$ を $\mathfrak{F}$ の上の有限加法的測度とする. このとき
    \begin{enumerate}[(i)]
        \item 任意の $A \subset X$ に対してたかだか可算無限個の集合 $E_n \in \mathfrak{F}$ で $A$ を覆い
        \begin{equation}
            \Gamma(A) = \inf \sum_{n = 1}^\infty m(E_n)
        \end{equation}
        と定義すると, $\Gamma$ は外測度である. 
        \item 特に, $m$ が $\mathfrak{F}$ の上で完全加法的ならば $E \in \mathfrak{F}$ に対しては $\Gamma(E) = m(E)$ となる. 
    \end{enumerate}
\end{theo}

\begin{defi}
    (Lebesgue外測度) \\
    $R^N$ において $f_v(\lambda) = \lambda (v = 1, \dots, N)$ として構成した外測度を $\mu^*(A)$ と書き {\bf Lebesgue}\textgt{外測度}という. 
    この場合は $I = (a_1, b_1] \times \cdots \times (a_N, b_N]$ に対して
    \begin{equation}
        \mu^*(I) = \prod_{v = 1}^N (b_v - a_v)
    \end{equation}
    となる. 
\end{defi}

\begin{theo}
    任意の $A \subset X$ に対して
    \begin{equation}
        \Gamma(A) \leq \Gamma(A \cap E) + \Gamma(A \cap E^c)
    \end{equation}
    が成り立つ. 
\end{theo}

\begin{defi} \label{gamma_measurable}
    ($\Gamma$-可測) \\
    空間 $X$ に外測度 $\Gamma$ が定義されているとする. 任意の $A \subset X$ に対して
    \begin{equation}
        \Gamma(A) = \Gamma(A \cap E) + \Gamma(A \cap E^c)
    \end{equation}
    を満たすとき, 集合 $E \subset X$ は ({\bf Carathéodory}\textgt{の意味で})\textgt{可測}または \textgt{$\Gamma$-可測}であるという. 
\end{defi}
\begin{theo}
    定義\ref{gamma_measurable}の条件は次の条件と同等である. 任意の $A_1 \subset E$ と任意の $A_2 \subset E^c$ に対して
    \begin{equation}
        \Gamma(A_1 + A_2) = \Gamma(A_1) + \Gamma(A_2)
    \end{equation}
\end{theo}

\begin{theo}
    $\Gamma$-可測集合の全体を $\mathfrak{M}_\Gamma$ と書く. すると以下が成り立つ. 
    \begin{enumerate}[(i)]
        \item $E \in \mathfrak{M}_\Gamma \Rightarrow E^c \in \mathfrak{M}_\Gamma$
        \item $\Gamma(E) = 0 \Rightarrow E \in \mathfrak{M}_\Gamma$, 従って特に $\emptyset \in \mathfrak{M}_\Gamma$
    \end{enumerate}
\end{theo}

\begin{defi}
    (零集合) \\
    $\Gamma(E) = 0$ なる集合を\textgt{零集合}という. 
\end{defi}

\begin{theo}
    定理\ref{outer_measure}の方法で構成された外測度 $\Gamma$ については $\mathfrak{F} \subset \mathfrak{M}_\Gamma$ が成立する. つまり有限加法族は $\Gamma$-可測である. 
\end{theo}

\begin{theo}
    $\displaystyle E_k \in \mathfrak{M}_\Gamma (k = 1, 2, \dots), E_j \cap E_k = \emptyset (j \neq k), S = \sum_{k = 1}^\infty E_k$ ならば
    \begin{equation}
        S \in \mathfrak{M}_\Gamma, \Gamma(S) = \sum_{k = 1}^\infty \Gamma(E_k)
    \end{equation}
\end{theo}

\begin{theo}
    $E, F \in \mathfrak{M}_\Gamma$ ならば $E - F \in \mathfrak{M}_\Gamma, E \cap F \in \mathfrak{M}_\Gamma$. 
\end{theo}

\begin{corr}
    $E_k \in \mathfrak{M}_\Gamma (k = 1, \dots, n)$ ならば $\displaystyle \bigcup_{k = 1}^n E_k \in \mathfrak{M}_\Gamma, \bigcap_{k = 1}^n E_k \in \mathfrak{M}_\Gamma$. 
\end{corr}

\begin{theo}
    $E_n \in \mathfrak{M}_\Gamma (n = 1, 2, \dots)$ ならば $\displaystyle \bigcup_{n = 1}^\infty E_n \in \mathfrak{M}_\Gamma$. 
\end{theo}

\section{測度}

\begin{defi}
    (測度) \\
    空間 $X$ の部分集合の族 $\mathfrak{B}$ があって
    \begin{enumerate}[(i)]
        \item $\emptyset \in \mathfrak{B}$
        \item $E \in \mathfrak{B} \Rightarrow E^c \in \mathfrak{B}$
        \item $\displaystyle E_n \in \mathfrak{B} (n = 1, 2, \dots) \Rightarrow \bigcup_{n = 1}^\infty E_n \in \mathfrak{B}$
    \end{enumerate}
    なる三つの条件を満たすとき, $\mathfrak{B}$ を\textgt{完全加法族, 可算加法族, }{\bf $\boldsymbol \sigma$-}\textgt{加法族}, または単に\textgt{加法族}という. 
\end{defi}

\begin{rema}
    有限加法族の場合と同様にして, 
    \begin{enumerate}[(i)]
        \item $X \in \mathfrak{B}$
        \item $\mathfrak{B}$ に属する集合の和, 差, 交わりを作る操作を高々可算無限回行って得られる集合は $\mathfrak{B}$ に属する. 
    \end{enumerate}
    が成り立つ. 
\end{rema}

\begin{defi}
    (測度) \\
    空間 $X$ とその部分集合の $\sigma$-加法族 $\mathfrak{B}$ があって, $\mathfrak{B}$-集合関数 $\mu(A)$ が
    \begin{enumerate}[(i)]
        \item (非負性) $0 \leq \mu(A) \leq \infty, \mu(\emptyset) = 0 $
        \item (完全加法性) $\displaystyle A_n \in \mathfrak{B} (n = 1, 2, \dots), A_j \cap A_k = \emptyset (j \neq k) \Rightarrow \mu\left( \sum_{n = 1}^\infty A_n \right) = \sum_{n = 1}^\infty \mu(A_n)$
    \end{enumerate}
    を満たすとき, $\mu$ を $\mathfrak{B}$ で定義された\textgt{測度}という. 
\end{defi}
\begin{rema}
    上二つの性質から
    \begin{enumerate}[(i)]
        \item (単調性) $A, B \in \mathfrak{B}, A \subset B \Rightarrow \mu(A) \leq \mu(B)$, 特に $\mu(B) < \infty \Rightarrow \mu(A - B) = \mu(A) - \mu(B)$
        \item (劣加法性) $\displaystyle A_1, \dots, A_n \in \mathfrak{B} \Rightarrow \mu\left( \bigcup_{j = 1}^n A_j \right) \leq \sum_{j = 1}^n \mu(A_j)$
    \end{enumerate}
    が成り立つ. 
\end{rema}

\begin{defi}
    (測度空間) \\
    空間 $X$ にその部分集合の $\sigma$-加法族 $\mathfrak{B}$ と $\mathfrak{B}$ で定義された測度 $\mu$ を組み合わせて考えたものを\textgt{測度空間}といい, $(X, \mathfrak{B}, \mu)$ または $X(\mathfrak{B}, \mu)$ と書く. 
\end{defi}

\begin{theo}
    $\Gamma$ を空間 $X$ で定義された外測度とすると, $\Gamma$-可測集合の全体 $\mathfrak{M}_\Gamma$ は $\sigma$-加法族をなし, $\Gamma$ は $\mathfrak{M}_\Gamma$ の上で定義された測度である. 
\end{theo}

\begin{theo}
    $A_n \in \mathfrak{B}$ とする. 
    \begin{enumerate}[(i)]
        \item 集合列 $\{A_n\}$ が単調増加のとき, または単調減少で $\mu(A_1) < \infty$ のときは
        \begin{equation}
            \mu(\lim_{n \to \infty} A_n) = \lim_{n \to \infty} \mu(A_n)
        \end{equation}
        \item 一般には
        \begin{align}
            \mu (\varliminf_{n \to \infty} A_n) \leq \varliminf_{n \to \infty} \mu(A_n) \\
            \mu\left( \bigcup_{n = 1}^\infty A_n \right) < \infty ならば \mu(\varlimsup_{n \to \infty} A_n) \geq \varlimsup_{n \to \infty} \mu(A_n) \\
            \mu\left( \bigcup_{n = 1}^\infty A_n \right) < \infty で \lim_{n \to \infty} A_n が存在するならば \mu(\lim_{n \to \infty} A_n) = \lim_{n \to \infty} \mu(A_n)
        \end{align}
    \end{enumerate}
\end{theo}



\end{document}